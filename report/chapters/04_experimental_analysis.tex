\section{Experimental analysis}
\label{sec:experimental_analysis}
    \subsection{Task}
      \label{sec:task}
        As mentioned in the introduction, the task consists in classifying sentences
        based on the sentiment expressed by them. after the training of the models,
        the desired output is a class, in the range from 1 to 5 stars in the first place,
        and in the range [0, 2] in the second place, where 0 is negative, 1 is neutral and
        2 is positive. \\
        
        Furthermore, such task is performed with the purpoose of not only gather 
        metrics about the generalization capabilities of the architectures implemented,
        but also to compare the performance of the two.
      

    \subsection{Experimental settings}
    Describe all the relevant aspects of the experimental setup used in your 
    experiment (e.g., how you performed model selection for fine-tuning of 
    hyper-parameters, all details regarding the learning / fine-tuning of your 
    model, etc.)
    \subsection{Results} 
    Provide results (figures and tables with mounerical results should go here). 
    Provide insights and comments on the achieved results (also comparatively with literature). Possible ablation studies go here.