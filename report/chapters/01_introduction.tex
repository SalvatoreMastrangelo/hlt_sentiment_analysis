\section{Introduction}
\label{sec:introduction}
    \subsection{Nature of sentiment analysis}
    \label{subsec:nature_of_sentiment_analysis}
        Sentiment Analysis is a subfield of Natural Language Processing (NLP) that focuses on identifying
        and classifying subjective information in text data. In particular, it allows to cathegorize the
        sentiment expressed by pieces of text, such as reviews, comments, or social media posts, into 
        predefined classes, that can be whether:
        \begin{itemize}
            \item favorable or unfavorable opinions towards a product, service, or entity,
            \item expressed emotions such as joy, anger, sadness, or fear,
            \item opinions about specific aspects or features of a product or service (aspect-based sentiment).
        \end{itemize}
        Sentiment analysis is widely used in various domains, including marketing, customer service, 
        and social media, to gain insights into public opinion or provide control over allowed content,
        like in the case of hate speech detection.

    \subsection{Approaches to Sentiment Analysis}
    \label{subsec:approaches_to_sentiment_analysis}
        The methods used for sentiment analysis can be broadly categorized into: 
        \begin{itemize}
            \item \textbf{Rule-based systems}, which use lexicons and pattern-based approaches, typically hand-crafted,
                    that require large efforts to develop and mantain \citep{gupta2024comprehensivestudysentimentanalysis}.
            \item \textbf{Feature engineering and Machine Learning}, that is based on extraction of features as
                    bag-of-words, n-grams, or word embeddings, followed by machine learning classifiers \citep{gupta2024comprehensivestudysentimentanalysis}.
        \end{itemize}
        In particular, machine learning models have gained in recent years a lot of attention, as recurrent models
        like Long Short-Term Memory (LSTM) networks have shown effective capabilities in capturing relations among
        distant words in a sentence \citep{staudemeyer2019understandinglstmtutorial}, and Transformers like
        BERT \citep{devlin2019bert} and its variants have shown state-of-the-art performance in many NLP tasks.

    \subsection{The project}
    \label{subsec:the_project}
        In this project, the focus is posed on implementation and evaluation two models, one based on LSTM-RNN 
        and one based on RoBERTa \citep{liu2019robertarobustlyoptimizedbert}, a variant of BERT \citep{devlin2019bert},
        that are able to classify the sentiment of hotel reviews, and return a score from 1 to 5 stars, as well
        as a ternary classification of the sentiment expressed, that can be positive, negative or 
        neutral.